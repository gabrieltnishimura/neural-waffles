% ---
% Arquivo com a introdução da Tese de Mestrado do aluno
% Gabriel Takaoka Nishimura da Escola Politécnica da Universidade de São Paulo
% ---
	\chapter*[Introdução]{Introdução}
	\addcontentsline{toc}{chapter}{Introdução}

	Na área da Robótica, a Localização e Mapeamento Simultâneo (SLAM) se refere ao mapeamento de um ambiente desconhecido e localização de um robô nesse local - ambos processos realizados de forma simultânea - por meio de sensores.
	
	A adição do contexto de localização ao robô (ou sistema) tem uma gama de aplicações, como carros autônomos, veículos aéreos não tripulados, robôs domésticos, agricultores e até dentro do corpo humano \cite{bao2014simultaneous}. Os usos citados tem um fator em comum: esses sistemas são predominantemente autônomos, ou seja, funcionam sem o auxílio de um operador.
	
	Entretanto, para que sua operação seja verdadeiramente autônoma, é necessário que a capacidade de localização e mapeamento seja integrada a seu sistema \cite{nikolic2014synchronized}. O robô ainda deve considerar controle, planejamento de caminho e tomada de decisão como processos imprescindíveis para seu funcionamento. Esse paralelismo é ainda mais restringido pelo poder computacional e energia disponível ao sistema.
	
	A literatura apresenta diversas formas de tentar resolver a Localização e Mapeamento Simultâneo em tempo real. Uma das soluções comumente utilizadas por carros autônomos é o sensor chamado LIDAR, que escaneia seu ambiente por meio da emissão de luz e análise de sua reflexão. Ele é muito utilizado devido a sua alta taxa de amostragem e precisão \cite{wolcott2017robust}. Seu custo, no entanto, é muito proibitivo, dificultando sua pesquisa. Uma alternativa mais barata ao LIDAR é o sensor RGB-D \cite{sturm2012benchmark}, que é basicamente uma câmera que identifica cor e distância dos pontos gravados. Sua precisão não é tão alta, nem ele funciona com ausência de luz, mas é objeto de pesquisa frequentemente visto - devido a seu baixo custo e grande comunidade. 
	
	O sensor RGB-D gera como saída uma Nuvem de Pontos (NP), que, dependendo da análise aplicada, torna possível a resolução do problema SLAM tridimensional \cite{henry2012rgb}. 	
	
	A NP se refere a um conjunto de pontos no sistema de coordenadas do sensor, com intuito de representar a face exterior do objeto escaneado. Com certos algoritmos, como EKF-SLAM ou FastSLAM \cite{durrant2006simultaneous}, é possível posicionar o robô dentro de um sistemas de coordenadas tridimensional independente do criado pelo sensor. A solução de uma CP é atualmente aprimorada por um conjunto de empresas (como a Intel, Toyota, nVIDIA), que disponibiliza uma biblioteca para resolver problemas de nuvem de pontos em três dimensões \cite{Rusu_ICRA2011_PCL}. Mas mesmo com a disponibilidade da biblioteca, esses problemas computacionais requerem um processamento com alta vazão para solucioná-los em tempo real \cite{clipp2010parallel}.
		
	Para resolver o problema de desempenho tanto em software quando em hardware, existem algumas abordagens tomadas, como por exemplo o uso de Unidade de Processamento Gráfico de Propósito Geral (GPGPU) \cite{lee2012gpu} e o uso de FPGAs \cite{nikolic2014synchronized}. 
	
	No caso de GPGPUs, existe um alto consumo de energia e de processamento, pois essas placas gráficas são feitas para processar um volume de dados de Gigabits/s devido à sua arquitetura com alto nível de paralelismo. As suas desvantagens são o alto consumo de energia e o fato de que não são específicas para resolver nuvens de pontos.
	
	FPGAs são utilizadas para atender a restrições de tempo real \cite{tertei2014fpga} de aplicações atuais com um algoritmo altamente específico para o problema. Com seu consumo energético reduzido e sua poderosa capacidade de processamento paralelo, as FPGAs se tornam um hardware interessante para implementar em sistemas embarcados complexos com restrições energéticas.
	
	É comum a área de engenharia se associar com outras áreas de conhecimento. É o caso desse projeto, que tem como aplicação a agricultura. O robô desenvolvido no projeto identificará ervas daninhas em uma plantação e as elimina através de um choque elétrico. Dessa forma, não é necessário o uso de agrotóxicos. Trabalhos anteriores referentes a agricultura também foram vistos \cite{cheein2011optimized}, mas não com o intuito de otimizar culturas orgânicas.
	
	\chapter{Objetivos}
	
	O objetivo desse projeto é a realização do estudo de técnicas e algoritmos para obter componentes relevantes a SLAM 3D (distância percorrida e ângulo de giro), analisando Núvens de Pontos geradas por sensores RGB-D em co-processador usando arquitetura FPGA. Por fim, o trabalho deve criar um sistema SLAM 3D com baixo consumo energético e computacional - utilizando as técnicas estudadas anteriormente - para implementação e testes em ambiente físico com robô.
	
	Esse robô deverá precisamente identificar a posição de uma erva daninha e eliminá-la por meio de um choque elétrico. 	

	%\section*{Motivação}\label{sec-motivacao}