% ---
% Arquivo com a introdução da Tese de Mestrado do aluno
% Gabriel Takaoka Nishimura da Escola Politécnica da Universidade de São Paulo
% ---
	\chapter*[Introdução]{Introdução}
	\addcontentsline{toc}{chapter}{Introdução}

	Na área da Robótica, a Localização e Mapeamento Simultâneo (LMS) se refere ao mapeamento de um ambiente desconhecido e localização de um robô nesse local - ambos processos realizados de forma simultânea -, por meio de sensores.
	
	A adição do contexto de localização ao robô (ou sistema) tem uma gama de aplicações, como carros autônomos, veículos aéreos não tripulados, robôs domésticos, agricultores e até dentro do corpo humano \cite{bao2014simultaneous}. Os usos citados tem um fator em comum: esses sistemas são predominantemente autônomos, ou seja, funcionam sem o auxílio de um operador.
	
	Entretanto, para que sua operação seja verdadeiramente autônoma, é necessário que a capacidade de localização e mapeamento seja integrada a seu sistema \cite{nikolic2014synchronized}. O robô ainda deve considerar controle, planejamento de caminho e tomada de decisão como processos imprescindíveis para seu funcionamento. Esse paralelismo é ainda mais restringido pelo poder computacional e energia disponível ao sistema.
	
	A literatura apresenta diversas formas de tentar resolver a Localização e Mapeamento Simultâneo em tempo real. Uma das soluções comumente utilizadas por carros autônomos é o sensor chamado LIDAR, que escaneia seu ambiente por meio da emissão de luz e análise de sua reflexão. Ele é muito utilizado devido a sua alta taxa de amostragem e precisão \cite{wolcott2017robust}. Seu custo, no entanto, é muito proibitivo, dificultando sua pesquisa. Uma alternativa mais barata ao LIDAR é o sensor RGB-D, que é basicamente uma câmera que identifica cor e distância dos pontos gravados. Sua precisão não é tão alta, nem ele funciona com ausência de luz, mas é objeto de pesquisa frequentemente visto - devido a seu baixo custo e grande comunidade. O sensor RGB-D gera como saída uma Núvem de Pontos (NP), que, dependendo da análise aplicada, torna possível a resolução do problema LMS tridimensional.
	
	A NP se refere a um conjunto de pontos no sistema de coordenadas do sensor, com intuito de representar a face exterior do objeto escaneado. Com certos algoritmos, como EKF-SLAM ou FastSLAM \cite{durrant2006simultaneous}, é possível posicionar o robô dentro de um sistemas de coordenadas tridimensional independente do criado pelo sensor. Mas esses problemas computacionais requerem um processamento com alta vazão para solucioná-los em tempo real.
		
	Para resolver o problema de desempenho tanto em software quando em hardware, FPGAs são utilizadas para atender a restrições de tempo real \cite{tertei2014fpga} de aplicações atuais. Com seu consumo energético reduzido e sua poderosa capacidade de processamento paralelo, as FPGAs se tornam um hardware interessante para implementar em sistemas embarcados complexos com restrições energéticas.

	
	
	parallel http://ieeexplore.ieee.org/document/5653696/
	http://ieeexplore.ieee.org/abstract/document/7836469/
	Kalmann http://scholarworks.csun.edu/handle/10211.3/176183
	Laser http://ieeexplore.ieee.org/abstract/document/1641929/
	bearing http://ieeexplore.ieee.org/abstract/document/1545393/
	6d slam http://onlinelibrary.wiley.com/doi/10.1002/rob.20209/full
	efficient http://ieeexplore.ieee.org/abstract/document/7832417/
	benchmark http://ieeexplore.ieee.org/abstract/document/6907054/
	evaluation http://ieeexplore.ieee.org/abstract/document/6225199/	
	
	\chapter*[Objetivos]{Objetivos}
	
	O objetivo desse projeto é a realização do estudo de técnicas e algoritmos para obter componentes relevantes a SLAM 3D (distância percorrida e ângulo de giro), analisando Núvens de Pontos geradas por sensores RGB-D em co-processador usando arquitetura FPGA. Por fim, o trabalho deve criar um sistema SLAM 3D com baixo consumo energético e computacional - utilizando as técnicas estudadas anteriormente - para implementação e testes em ambiente físico com robô.

	\begin{comment}
	\section*{Motivação}\label{sec-motivacao}
		
	Motivacao
	\cite{Rusu_ICRA2011_PCL}
	\end{comment}