% ---
% Arquivo com a metodologia da Tese de Mestrado do aluno
% Gabriel Takaoka Nishimura da Escola Politécnica da Universidade de São Paulo
% ---
	% ---
	\chapter{Metodologia}\label{cap-metodologia}
	% ---

	Com o intuito de implementar um sistema de navegação robótica utilizando SLAM 3D, propõe-se um método que consiste em duas etapas. A primeira etapa é a pesquisa bibliográfica, que consistirá no entendimento de algoritmos para nuvens de pontos.
	
	Para o desenvolvimento do co-processador, será necessário principalmente o uso de uma FPGA com o software de programação Altera Quartus. A nuvem de pontos será obtida com o periférico Kinect, e as salas usadas serão o Laboratório de Automação Agrícola e o Laboratório de Sistemas Digitais. Adicionalmente, o robô será adquirido por meio da professora Anna Reali.
	
	\chapter{Crongrama}\label{cap-cronograma}
	
	Para a execução do projeto de pesquisa, serão realizadas as atividades abaixo, de acordo com a \autoref{tab-cronograma}

	\begin{enumerate}
		\item Especificação do algoritmo para análise de nuvem de pontos em FPGA.
		\item Transformação da FPGA em co-processador do robô.
		\item Testes de funcionamento do algoritmo implementado no robô em ambiente real.
		\item Estudo do consumo energético e processamento computacional da FPGA.
		\item Escrita de artigo para publicação para divulgar os resultados obtidos.
		\item Escrita da monografia.
	\end{enumerate}

	\begin{table}[h]
		\centering
		\caption{Planejamento de atividades}
		\label{tab-cronograma}
		\begin{tabular}{lllrl}
			\multicolumn{1}{c}{Atividade / Semestre} & 1 & 2 & 3 & 4 \\ \hline
			1                                        & X &   &   &   \\
			2                                        &   & X &   &   \\
			3                                        &   &   & X &   \\
			4                                        &   &   & X &   \\
			5                                        &   &   &   & X \\
			6                                        &   &   &   & X
		\end{tabular}
	\end{table}