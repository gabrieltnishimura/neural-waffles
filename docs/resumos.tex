% ---
% Arquivo com o resumo da Tese de Mestrado do aluno
% Gabriel Takaoka Nishimura da Escola Politécnica da Universidade de São Paulo
% ---

% resumo em português
\setlength{\absparsep}{18pt} % ajusta o espaçamento dos parágrafos do resumo
\begin{resumo}
	Apresentação tema, popularidade SoC com duas pilhas, apesar disso (apenas avaliacoes comparativas, ao inves de cooperativas), mecanismo de cooperacao, analises numericas e testes para avaliar

	SLAM 3D tem várias aplicações, algumas mais baratas, como o RGB-D com kinect. Em sistemas 
	\vspace{\onelineskip}
	\noindent 
	\textbf{Palavras-chave}: Núvem de pontos, Localização e Mapeamento Simultâneos, RGB-D, FPGA
\end{resumo}

% resumo em inglês
\begin{comment}
\begin{resumo}[Abstract]
	\begin{otherlanguage*}{english}
		Resumo em ingles
		
		\vspace{\onelineskip}
		\noindent 
		\textbf{Keywords}: keyword 1.
	\end{otherlanguage*}
\end{resumo}
\end{comment}