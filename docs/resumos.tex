% ---
% Arquivo com o resumo da Tese de Mestrado do aluno
% Gabriel Takaoka Nishimura da Escola Politécnica da Universidade de São Paulo
% ---

% resumo em português
\setlength{\absparsep}{18pt} % ajusta o espaçamento dos parágrafos do resumo
\begin{resumo}
	Localização e Mapeamento Simultânea 3D (SLAM 3D) se refere a um conjunto de algoritmos e técinas aplicados para obter a posição atual de um sistema em tempo real. O SLAM 3D tem diversas aplicações, como carros autônomos, veículos aéreos não tripulados, robôs agricultores, e até atuadores dentro do corpo humano. No entanto, o cálculo da posição do sistema tem algoritmo muito custoso, com complexidade adicional necessitando processar esses dados em tempo real. O trabalho abaixo propõe a utilização de uma camera RGB-D (Kinect), para criar um co-processador que calcule a distância percorrida e ângulo de giro entre duas imagens dessa câmera. Esses parâmetros poderão ser utilizados posteriormente para a utilização de um robô que detecta e elimina ervas daninhas em uma plantação orgânica.
	\vspace{\onelineskip}
	\noindent 
	\textbf{Palavras-chave}: Nuvem de pontos, Localização e Mapeamento Simultâneos, RGB-D, FPGA, Agricultura
\end{resumo}

% resumo em inglês
\begin{comment}
\begin{resumo}[Abstract]
	\begin{otherlanguage*}{english}
		Resumo em ingles
		
		\vspace{\onelineskip}
		\noindent 
		\textbf{Keywords}: keyword 1.
	\end{otherlanguage*}
\end{resumo}
\end{comment}