% ---
% Arquivo com o resumo da Tese de Mestrado do aluno
% Gabriel Takaoka Nishimura da Escola Politécnica da Universidade de São Paulo
% ---

% resumo em português
\setlength{\absparsep}{18pt} % ajusta o espaçamento dos parágrafos do resumo
\begin{resumo}
	Localização e Mapeamento 3D (SLAM 3D) tem diversas aplicações, como carros autônomos, veículos aéreos não tripulados, robôs domésticos, agricultores e até dentro do corpo humano. Para implementá-la, existem diversos sensores que medem uma multitude de parâmetros. No trabalho, escolheu-se o Kinect, que é um sensor RGB-D, capaz de medir a distância entre pontos capturados, devido a existência de uma biblioteca para geração e manipulação de nuvem de pontos (CPL). Como essa biblioteca é computacionalmente exigente, será criado um co-processador para auxiliar o processamento da nuvem de pontos gerada pelo sensor RGB-D. A saída detectada deve informar o sistema da distância percorrida e do ângulo de giro, possibilitando o robô a detectar e eliminar ervas daninhas em uma plantação orgânica.
	
	\vspace{\onelineskip}
	\noindent 
	\textbf{Palavras-chave}: Nuvem de pontos, Localização e Mapeamento Simultâneos, RGB-D, FPGA, Agricultura
\end{resumo}

% resumo em inglês
\begin{comment}
\begin{resumo}[Abstract]
	\begin{otherlanguage*}{english}
		Resumo em ingles
		
		\vspace{\onelineskip}
		\noindent 
		\textbf{Keywords}: keyword 1.
	\end{otherlanguage*}
\end{resumo}
\end{comment}