% ------------------------------------------------------------------------
% ------------------------------------------------------------------------
% Tese de Mestrado da Escola Politécnica da USP
% Integrantes: Gabriel Takaoka Nishimura
% ------------------------------------------------------------------------
% ------------------------------------------------------------------------

\documentclass[
	12pt,				% tamanho da fonte
	openright,			% capítulos começam em pág ímpar
	oneside,			% para impressão em recto e verso. Oposto a oneside
	a4paper,			% tamanho do papel.
	hyphens,			% url longa nas referencias
	english,			% idioma adicional para hifenização
	brazil				% o último idioma é o principal do documento
]{abntex2}

% ---
%  PACOTES
% ---

% Pacotes básicos
\usepackage{lmodern}			% Usa a fonte Latin Modern
\usepackage[T1]{fontenc}		% Selecao de codigos de fonte.
\usepackage[utf8]{inputenc}		% Conversão automática dos acentos
\usepackage{lastpage}			% Usado pela Ficha catalográfica
\usepackage{indentfirst}		% Indenta o primeiro parágrafo de cada seção.
\usepackage{color}				% Controle das cores
\usepackage{amssymb}			% Less than or equal to
\usepackage{tikz}				% desenhos de graficos aux
\usepackage{pgfplots}			% desenhos de gráficos
\usepackage{graphicx}			% Inclusão de gráficos
\usepackage{subcaption}			% Subfigure
\usepackage{float}				% Gráficos
\usepackage{gensymb}			% degree
\usepackage{microtype} 			% para melhorias de justificação
\usepackage{lipsum}				% para geração de dummy text
\usepackage{listings}			% para listagem de código %
\usepackage{amsmath}			% para construção de sistemas lineares%


% Pacotes de citações
\usepackage[brazilian,hyperpageref]{backref}	 % Paginas com as citações
\usepackage[alf,abnt-url-package=url]{abntex2cite}	% Citações padrão ABNT

% ---
% CONFIGURAÇÕES DE PACOTES
% ---

% ---
% Configurações do pacote
\newenvironment{spmatrix}[1]{
	\def\mysubscript{#1}\mathop\bgroup\begin{pmatrix}
	}
{\end{pmatrix}\egroup_{\textstyle\mathstrut\mysubscript}}
% ---

% ---
% Configurações do pacote PGFPLOT
\pgfplotsset{compat=1.14}
\usetikzlibrary{patterns}
\usepgfplotslibrary{fillbetween}

% ---
% Configurações do pacote float
\newfloat{chart}{tbhp}{loc} %section
\floatname{chart}{Gráfico}
\floatstyle{plaintop}
\restylefloat{chart}
\newcommand{\listofcharts}{\listof{chart}{Lista de Gráficos}}
% ---
% Configurações do pacote backref
\renewcommand{\backrefpagesname}{Citado na(s) página(s):~}
\renewcommand{\backref}{}
\renewcommand*{\backrefalt}[4]{
	\ifcase #1 %
	Nenhuma citação no texto.%
	\or
	Citado na página #2.%
	\else
	Citado #1 vezes nas páginas #2.%
	\fi}%
% ---

% ---
% Informações de dados para CAPA e FOLHA DE ROSTO
% ---
\titulo{\textit{SLAM 3D e RGB-D}: Uma abordagem com FPGA}
\autor{Gabriel Takaoka Nishimura}
\local{São Paulo}
\data{2016}
\orientador{Prof. Dr. Bruno Carvalho de Albertini}
\instituicao{Universidade de São Paulo -  USP \par
			Escola Politécnica \par
			Engenharia Elétrica com ênfase em Computação}
\tipotrabalho{Mestrado}
\preambulo{Plano de Pesquisa apresentado ao Departamento de Engenharia de Computação e Sistemas 
			Digitais da Escola Politécnica da Universidade de São Paulo para processo de seleção 
			de Mestrado do segundo semestre de 2017}
%\preambulo{Tese de Mestrado apresentada ao Departamento de Engenharia de Computação e Sistemas
%	Digitais da Escola Politécnica da Universidade de São Paulo para obtenção do título de Mestre}
% ---

% ---
% Configurações de aparência do PDF final
\definecolor{blue}{RGB}{41,5,195} % cor azul

% informações do PDF
\makeatletter
\hypersetup{
	%pagebackref=true,
	pdftitle={\@title},
	pdfauthor={\@author},
	pdfsubject={\imprimirpreambulo},
	pdfcreator={LaTeX with abnTeX2},
	pdfkeywords={abnt}{latex}{abntex}{abntex2}{trabalho acadêmico},
	colorlinks=true,       		% false: boxed links; true: colored links
	linkcolor=blue,          	% color of internal links
	citecolor=blue,        		% color of links to bibliography
	filecolor=magenta,      		% color of file links
	urlcolor=blue,
	bookmarksdepth=4
}
\makeatother
% ---

% ---
% Espaçamentos entre linhas e parágrafos
% ---
\setlength{\parindent}{1.3cm} % O tamanho do parágrafo
\setlength{\parskip}{0.2cm}  % espacamento entre um paragrafo e outro

% ---
% compila o indice
% ---
\makeindex
% ---

% ----
% Início do documento
% ----
\begin{document}

	% Seleciona o idioma do documento
	\selectlanguage{brazil}

	% Retira espaço extra obsoleto entre as frases.
	\frenchspacing

	% ----------------------------------------------------------
	% ELEMENTOS PRÉ-TEXTUAIS
	% ----------------------------------------------------------
	% \pretextual

	% ---
	% Capa
	% ---
	%\imprimircapa
	% ---

	% ---
	% Folha de rosto
	% ---
	\imprimirfolhaderosto*
	% ---
	
	% ---
	% Ficha Catalográfica
	% Quando receber pdf da biblioteca, descomentar linhas abaixo
	% ---
	%\include{ficha-catalografica}
	% ---

	% \begin{fichacatalografica}
	%     \includepdf{fig_ficha_catalografica.pdf}
	% \end{fichacatalografica}

	% ---
	% TODO APROVACAO
	% Quando receber pdf com assinaturas, descomentar linha abaixo
	% ---
	% \includepdf{folhadeaprovacao_final.pdf}
	%

	% ---
	% Agradecimentos
	% ---
	%\begin{agradecimentos}
	%\end{agradecimentos}
	% ---

	% ---
	% Epígrafe
	% ---
	\begin{comment}
	\begin{epigrafe}
		\vspace*{\fill}
		\begin{flushright}
			\textit{``In a dark place we find ourselves,\\
				and a little more knowledge	lights our way.\\
				(Yoda, Episode 3: Revenge of the Sith)}
		\end{flushright}
	\end{epigrafe}
	\end{comment}
	% ---

	% ---
	% RESUMOS
	% ---
	% ---
% Arquivo com o resumo da Tese de Mestrado do aluno
% Gabriel Takaoka Nishimura da Escola Politécnica da Universidade de São Paulo
% ---

% resumo em português
\setlength{\absparsep}{18pt} % ajusta o espaçamento dos parágrafos do resumo
\begin{resumo}
	Resumo em português
	
	\vspace{\onelineskip}
	\noindent 
	\textbf{Palavras-chave}: palavra-chave 1.
\end{resumo}

% resumo em inglês
\begin{resumo}[Abstract]
	\begin{otherlanguage*}{english}
		Resumo em ingles
		
		\vspace{\onelineskip}
		\noindent 
		\textbf{Keywords}: keyword 1.
	\end{otherlanguage*}
\end{resumo}
	% ---
	

	% ---
	% inserir lista de ilustrações
	% ---
	%\pdfbookmark[0]{\listfigurename}{lof}
	%\listoffigures*
	%\cleardoublepage
	% ---

	% ---
	% inserir lista de gráficos
	% ---
	%\pdfbookmark[0]{\listtablename}{lot}
	%\listofcharts
	%\cleardoublepage
	% ---

	% ---
	% inserir lista de tabelas
	% ---
	%\pdfbookmark[0]{\listtablename}{lot}
	%\listoftables*
	%\cleardoublepage
	% ---

	% ---
	% inserir lista de abreviaturas e siglas
	% ---
	
	% ---

	% ---
	% inserir o sumario
	% ---
	\pdfbookmark[0]{\contentsname}{toc}
	\tableofcontents*
	\cleardoublepage
	% ---

	% ----------------------------------------------------------
	% ELEMENTOS TEXTUAIS
	% ----------------------------------------------------------
	\textual

	% ----------------------------------------------------------
	% Introdução
	% ----------------------------------------------------------
	% ---
% Arquivo com a introdução da Tese de Mestrado do aluno
% Gabriel Takaoka Nishimura da Escola Politécnica da Universidade de São Paulo
% ---
	\chapter*[Introdução]{Introdução}
	\addcontentsline{toc}{chapter}{Introdução}

	Na área da Robótica, a Localização e Mapeamento Simultâneo (LMS) se refere ao mapeamento de um ambiente desconhecido e localização de um robô nesse local - ambos processos realizados de forma simultânea -, por meio de sensores.
	
	A adição do contexto de localização ao robô (ou sistema) tem uma gama de aplicações, como carros autônomos, veículos aéreos não tripulados, robôs domésticos, agricultores e até dentro do corpo humano \cite{bao2014simultaneous}. Os usos citados tem um fator em comum: esses sistemas são predominantemente autônomos, ou seja, funcionam sem o auxílio de um operador.
	
	Entretanto, para que sua operação seja verdadeiramente autônoma, é necessário que a capacidade de localização e mapeamento seja integrada a seu sistema \cite{nikolic2014synchronized}. O robô ainda deve considerar controle, planejamento de caminho e tomada de decisão como processos imprescindíveis para seu funcionamento. Esse paralelismo é ainda mais restringido pelo poder computacional e energia disponível ao sistema.
	
	A literatura apresenta diversas formas de tentar resolver a Localização e Mapeamento Simultâneo em tempo real. Uma das soluções comumente utilizadas por carros autônomos é o sensor chamado LIDAR, que escaneia seu ambiente por meio da emissão de luz e análise de sua reflexão. Ele é muito utilizado devido a sua alta taxa de amostragem e precisão \cite{wolcott2017robust}. Seu custo, no entanto, é muito proibitivo, dificultando sua pesquisa. Uma alternativa mais barata ao LIDAR é o sensor RGB-D, que é basicamente uma câmera que identifica cor e distância dos pontos gravados. Sua precisão não é tão alta, nem ele funciona com ausência de luz, mas é objeto de pesquisa frequentemente visto - devido a seu baixo custo e grande comunidade. O sensor RGB-D gera como saída uma Núvem de Pontos (NP), que, dependendo da análise aplicada, torna possível a resolução do problema LMS tridimensional.
	
	A NP se refere a um conjunto de pontos no sistema de coordenadas do sensor, com intuito de representar a face exterior do objeto escaneado. Com certos algoritmos, como EKF-SLAM ou FastSLAM \cite{durrant2006simultaneous}, é possível posicionar o robô dentro de um sistemas de coordenadas tridimensional independente do criado pelo sensor. Mas esses problemas computacionais requerem um processamento com alta vazão para solucioná-los em tempo real.
		
	Para resolver o problema de desempenho tanto em software quando em hardware, FPGAs são utilizadas para atender a restrições de tempo real \cite{tertei2014fpga} de aplicações atuais. Com seu consumo energético reduzido e sua poderosa capacidade de processamento paralelo, as FPGAs se tornam um hardware interessante para implementar em sistemas embarcados complexos com restrições energéticas.

	
	
	parallel http://ieeexplore.ieee.org/document/5653696/
	http://ieeexplore.ieee.org/abstract/document/7836469/
	Kalmann http://scholarworks.csun.edu/handle/10211.3/176183
	Laser http://ieeexplore.ieee.org/abstract/document/1641929/
	bearing http://ieeexplore.ieee.org/abstract/document/1545393/
	6d slam http://onlinelibrary.wiley.com/doi/10.1002/rob.20209/full
	efficient http://ieeexplore.ieee.org/abstract/document/7832417/
	benchmark http://ieeexplore.ieee.org/abstract/document/6907054/
	evaluation http://ieeexplore.ieee.org/abstract/document/6225199/	
	
	\chapter*[Objetivos]{Objetivos}
	
	O objetivo desse projeto é a realização do estudo de técnicas e algoritmos para obter componentes relevantes a SLAM 3D (distância percorrida e ângulo de giro), analisando Núvens de Pontos geradas por sensores RGB-D em co-processador usando arquitetura FPGA. Por fim, o trabalho deve criar um sistema SLAM 3D com baixo consumo energético e computacional - utilizando as técnicas estudadas anteriormente - para implementação e testes em ambiente físico com robô.

	\begin{comment}
	\section*{Motivação}\label{sec-motivacao}
		
	Motivacao
	\cite{Rusu_ICRA2011_PCL}
	\end{comment}
	% ----------------------------------------------------------
	
	% ----------------------------------------------------------
	% Revisões Bibliográficas
	% ----------------------------------------------------------
	%% ---
% Arquivo com a revisão bibliográfica da Tese de Mestrado do aluno
% Gabriel Takaoka Nishimura da Escola Politécnica da Universidade de São Paulo
% ---
	\chapter{Revisão Bibliográfica}\label{cap-revisao-bibliografica}
	
	Revisão bibliográfica
	
	% ----------------------------------------------------------

	% ----------------------------------------------------------
	% Metodologia
	% ----------------------------------------------------------
	% ---
% Arquivo com a metodologia da Tese de Mestrado do aluno
% Gabriel Takaoka Nishimura da Escola Politécnica da Universidade de São Paulo
% ---
	% ---
	\chapter{Metodologia}\label{cap-metodologia}
	% ---

	Com o intuito de implementar um sistema de navegação robótica utilizando SLAM 3D, propõe-se um método que consiste em duas etapas. A primeira etapa 
	
	
	Uma placa FPGA será utilizada para 
	
	
	\chapter{Crongrama}\label{cap-cronograma}
	
	
	% ----------------------------------------------------------
	
	% ----------------------------------------------------------
	% Execução
	% ----------------------------------------------------------
	%% ---
% Arquivo com a execução da Tese de Mestrado do aluno
% Gabriel Takaoka Nishimura da Escola Politécnica da Universidade de São Paulo
% ---
	\chapter{Execução}\label{cap-execucao}
	
	Execução aqui.
	% ----------------------------------------------------------

	% ----------------------------------------------------------
	% Prepara pdf para iniciar o bookmark na raiz
	% ----------------------------------------------------------
	\phantompart

	% ----------------------------------------------------------
	% Conclusão
	% ----------------------------------------------------------
	%% ---
% Arquivo com a conclusão da Tese de Mestrado do aluno
% Gabriel Takaoka Nishimura da Escola Politécnica da Universidade de São Paulo
% ---
	\chapter{Conclusão}\label{cap-conclusao}

	Conclusao aqui.
	% ----------------------------------------------------------

	% ----------------------------------------------------------
	% ELEMENTOS PÓS-TEXTUAIS
	% ----------------------------------------------------------
	\postextual
	% ----------------------------------------------------------

	% ----------------------------------------------------------
	% Referências bibliográficas
	% ----------------------------------------------------------
	\bibliography{referencias}

	% ----------------------------------------------------------
	% Glossário
	% ----------------------------------------------------------
	%\glossary
	% ----------------------------------------------------------

	% ----------------------------------------------------------
	% Apêndices
	% ----------------------------------------------------------
	%\begin{apendicesenv}
	%	\partapendices % pagina indicando inicio dos apendices
	%	% ---
% Arquivo com os apêndices do Trabalho de Conclusão de Curso dos alunos
% Gabriel Takaoka Nishimura, Felippe Demarqui Ramos e Vivian Kimie Isuyama 
% da Escola Politécnica da Universidade de São Paulo
% ---
\chapter{Diagramas da Arquitetura}

\begin{figure}[h]
	\caption{\label{fig_sindrome_arq}Arquitetura do módulo de cálculo das síndromes.}
	\centering
	\includegraphics[width=1.0\textwidth, trim={1.5cm 1.2cm 1.5cm 3cm}, clip]{RS/SindromeRTL.pdf}
	\legend{Fonte: Autores.}
\end{figure}

\begin{figure}[h]
	\caption{\label{fig_berlekamp_arq}Arquitetura do módulo de Berlekamp-Massey.}
	\centering
	\includegraphics[width=1.0\textwidth, trim={0 1.2cm 0 3cm}, clip]{RS/BerlekampRTL.pdf}
	\legend{Fonte: Autores.}
\end{figure}

\begin{figure}[h]
	\caption{\label{fig_chienloc_arq}Arquitetura do módulo de busca de Chien: localização de erros.}
	\centering
	\includegraphics[width=1.0\textwidth, trim={0 8cm 0 9cm}, clip]{RS/ChienLocationRTL.pdf}
	\legend{Fonte: Autores.}
\end{figure}

\begin{figure}[h]
	\caption{\label{fig_chienval_arq}Arquitetura do módulo de busca de Chien: valores de erros.}
	\centering
		\includegraphics[width=1.0\textwidth, trim={0 8cm 0 9cm}, clip]{RS/ChienValueRTL.pdf}
	\legend{Fonte: Autores.}
\end{figure}

\begin{figure}[h]
	\caption{\label{figure:interleaver-rtl}Arquitetura do módulo de \textit{Interleaver}.}
	\centering
	\includegraphics[width=1\textwidth, trim={0 12cm 0 12.5cm}, clip]{interleaver/rtl.pdf}
	\legend{Fonte: Autores.}
\end{figure}
\begin{figure}[h]
	\caption{\label{figure:sync-rtl}Arquitetura do módulo de Sincronização.}
	\centering
	\includegraphics[width=0.5\textwidth, trim={0 2cm 0 3cm}, clip]{sync/rtl.pdf}
	\legend{Fonte: Autores.}
\end{figure}
\begin{figure}[h]
	\caption{\label{figure:manchester-decoder-rtl}Arquitetura do módulo de Decodificação Manchester.}
	\centering
	\includegraphics[width=0.85\textwidth, trim={0 7.7cm 0 8.5cm}, clip]{manchester/decoder-rtl.pdf}
	\legend{Fonte: Autores.}
\end{figure}

\begin{figure}[h]
	\caption{\label{figure:viterbi-rtl}Arquitetura do módulo de Decodificação Viterbi.}
	\centering
	\includegraphics[width=1\textwidth, trim={0 7.7cm 0 8.5cm}, clip]{viterbi/rtl.pdf}
	\legend{Fonte: Autores.}
\end{figure}
	%\end{apendicesenv}
	% ----------------------------------------------------------

	% ----------------------------------------------------------
	% Anexos - SEM ANEXOS
	% ----------------------------------------------------------
	%\begin{anexosenv}
		%\partanexos	% Imprime uma página indicando o início dos anexos
		%\include{anexos}
	%\end{anexosenv}
	% ----------------------------------------------------------

	%---------------------------------------------------------------------
	% INDICE REMISSIVO
	%---------------------------------------------------------------------
	\phantompart
	\printindex
	%---------------------------------------------------------------------

\end{document}
